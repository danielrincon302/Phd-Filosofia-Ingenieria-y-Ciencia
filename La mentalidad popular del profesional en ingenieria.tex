\documentclass[letterpaper,12pt]{article}

\usepackage[spanish]{babel}
\usepackage[utf8]{inputenc}
\usepackage{graphicx}
\PassOptionsToPackage{hyphens}{url}\usepackage{hyperref}
\DeclareGraphicsExtensions{.jpg,.pdf,.mps,.png}

% Márgenes
\usepackage[top=2cm, left=2cm, right=2cm, bottom=2cm]{geometry}

% Interlineado
\linespread{1.5}

\usepackage{hyperref}

% Times Roman
\usepackage{mathptmx}

\usepackage{natbib}
\setcitestyle{super}

\usepackage{blindtext}

\linespread{1.0}\selectfont

\begin{document}

\title{La mentalidad popular del profesional en ingeniería empleando la filosofía de la ciencia como forma de estimular el pensamiento científico}
\author{\normalsize{Cesar Daniel Rincón Brito}}
\date{\normalsize{\today}}
\date{Junio 4, 2022}
\maketitle

\thispagestyle{empty}
\pagestyle{empty}

Hipócrates de Cos nacido en el año 460 a. e. c. en Tesalia, se le rememora por alejar la superstición de la medicina y conducirla a la ciencia. En un aforismo muy reconocido Hipócrates menciona: <<La humanidad cree que la epilepsia es divina, simplemente porque no tienen la capacidad de entenderla. Pero si nombraran divino a todo lo que no pueden comprender, habría una infinidad de cosas divinas>>.  A partir de esta introducción del método científico y el avance del conocimiento de la medicina se empezó a dejar de presumir dichos eventos a la intervención divina, por lo tanto, aumentando el tiempo y calidad de vida de miles de personas en este campo (Carl Sagan y Druyan, 1997).
\newline %line break

Para la ingeniería la historia de los instrumentos ha generado un efecto positivo para el progreso científico, ahora bien, nos preguntamos cuál es la causa del estancamiento o las malas prácticas de los profesionales, se puede evidenciar epistemológicamente la ciencia como un surgimiento que parte de cuantiosos investigadores ligados a las creencias y la magia. Caso contemporáneo, se puede apreciar en centros educativos de alta calidad que aún persiste la falta de escepticismo en la ciencia u otro tipo de programa, como ejemplo de ello observamos universidades como la Javeriana de Cali, Colombia que su distintivo se relaciona a un sector religioso entre los cientos existentes, y desde sus medios de comunicación alcanzando arraigar este imperativo de la moral, empero estas acciones no estás ligadas solo a este sector, también encontramos propuestas pseudocientíficas en la salud, caso de la polémica homeopatía, ofrecidas asimismo por la universidad nacional de Colombia como medicina alternativa, de tal modo se identifica a la Universidad Javeriana de Cali con diplomados como el siguiente: <<Diplomado internacional Homeopatía clásica - En convenio con el International Academy of Classical Homeopathy - (Agosto 2022)>>.
\newline %line break

Como un argumento de autoridad se utiliza el nombre de George Vithoulkas fundador de International Academy of Classical Homeopathy en el año 1995, institución que recientemente posee convenio con la universidad Javeriana de Cali, es otro reflejo del profesional en ingeniería que utiliza las ciencias ocultas ofreciendo tratamientos en la salud incluso sin contar con certificaciones en medicina, donde afirma que esta no previene ni cura las enfermedades, no obstante, se menciona en su biografía la Universidad Aegean que en el año 2007 ofrecía un programa con el nombre de posgrado en Homeopatía clásica sostiene que este fue suspendido en 2013.
\newline %line break

Cabe destacar que las creencias o prácticas seudocientíficas afectan a todas las áreas del conocimiento, la astronomía tiene la astrología, la medicina tiene la homeopatía, la psicología al psicoanálisis, la psiquiatría tiene la parasicología, la química la alquimia, desencadenando en el campo de las ciencias aplicadas el diseño de nuevas tecnologías dañinas para la sociedad, evidenciando ingenieros aprobando el uso de máquinas como el polígrafo, el rechazo a una vacuna o aceptar los conceptos de astronomía erróneos de culturas muy antiguas.
\newline %line break
 %line break
En un paso por el siglo XVII, uno de gran peso, Isaac Newton que <<veía el universo como un criptograma dispuesto por el todo poderoso>> (Grayling A, 2017), es decir que ni siquiera un científico, investigador o ingeniero con gran habilidad, pueda distinguir entre el conocimiento auténtico y no auténtico. 
\newline %line break

El neurólogo Oliver Sacks en su libro Alucinaciones (2013), acerca de la privación sensorial escribió: 
\begin{quote}
El cerebro necesita no sólo recibir percepciones, sino también un cambio perceptivo, y la ausencia de cualquier cambio podría provocar no sólo lapsus de despertar y atención, sino también aberraciones perceptivas (p.49).
\end{quote} 
\newline %line break

Ante estos sucesos, que se han visto a lo largo de este ensayo, y como profesionales en ingeniería con la responsabilidad mutua de transmitir el conocimiento a futuras generaciones es necesario fortalecer el proceso de aprendizaje partiendo del uso del método científico para fomentar la investigación formativa en las ingenierías.
\newline %line break

En la filosofía de la ciencia encontramos el camino a las prácticas aceptadas por la comunidad científica, es el examen a las ciencias básicas o aplicadas, de esta manera se solidifican ciertos atributos característicos modernos como la ontología, semántica, epistemología, ética, soportados por los aportes de las filosofías de la tecnología como el lenguaje, la matemática y la lógica respondiendo a diferentes preguntas aplicando fundamentos teóricos o fácticos en el campo de las ciencias de la computación de forma racional y objetiva analizando los fenómenos sociales, naturales y sus causas tomando como base la observación y experimentación para resolver problemas del mundo real.




\begin{thebibliography}{}
    \bibitem{libro}
        Bunge, Mario (2014), \textbf{La ciencia, su método y su filosofía}, Penguin Random House Grupo Editorial, Argentina.
    \bibitem{libro}
        Grayling, A (2017), \textbf{La era del ingenio}, Editorial Ariel, Barcelona, España.
    \bibitem{libro}
        Sacks, Oliver (2013), \textbf{Alucinaciones}, Editorial Anagrama colección argumentos, Barcelona, España.
    \bibitem{libro}
        Sagan, Carl. y Druyan A. (1997), \textbf{El mundo y sus demonios}, Ediciones B, Barcelona, España.
    \bibitem{link}
        ``Pontificia Universidad Javeriana. - Diplomado internacional Homeopatía clásica - En convenio con el International Academy of Classical Homeopathy.'' (en línea), Javeriana Cali, consultado el 04-06-2022. \\
        \url{https://www.javerianacali.edu.co/educacion-continua/diplomado-internacional-homeopatia-clasica-en-convenio-con-el-international-0} 
    \bibitem{wiki}
        ``Wikipedia contributors. (2021, diciembre 19).'' (en línea), En Wikipedia, consultado el 04-06-2022. \\
        \url{https://en.wikipedia.org/w/index.php?title=University_of_the_Aegean&oldid=1061049226}
\end{thebibliography}

\end{document}